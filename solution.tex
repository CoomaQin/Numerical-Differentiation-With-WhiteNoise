\documentclass{article}
\usepackage{graphicx}
\usepackage{amsmath}
\usepackage{listings}
\usepackage[framed,numbered,autolinebreaks,useliterate]{mcode}
\author{Huixiong Qin-2015141211046}
\title{Numerical Experiment}

\begin{document}
\maketitle

\section{}
The results are listed below(retain 4 decimal places).\\
$$h=\frac1{2^1}$$
\begin{center}
\begin{tabular}{|c|c|c|}
  \hline
  % after \\: \hline or \cline{col1-col2} \cline{col3-col4} ...
        & approx$f''(x)$ &error\\
  \hline
  $x_0$ & 2.7448  &0.1926\\
  $x_1$ & 2.1467  &0.2595\\
  $x_2$ & 1.0446 &0.3339\\
  $x_3$ & -0.7016 &0.4123\\
  $x_4$ & -3.2379 &0.4889\\
  $x_5$ & -6.7053 &0.5554\\
  $x_6$ & -11.2230 &0.6006\\
  $x_7$ & -16.8658 &0.6090\\
  $x_8$ & -23.6361 &0.5622\\
  $x_9$ & -31.4265 &0.4374\\
  $x_{10}$ & -39.9776 &0.2085\\
  \hline
\end{tabular}
\end{center}

$$h=\frac1{2^2}$$
\begin{center}
\begin{tabular}{|c|c|c|}
  \hline
  % after \\: \hline or \cline{col1-col2} \cline{col3-col4} ...
        & approx$f''(x)$ &error\\
  \hline
  $x_0$ & 2.8896  &0.0478\\
  $x_1$ & 2.3416  &0.0645\\
  $x_2$ & 1.2952 &0.0833\\
  $x_3$ & -0.3925 &0.1032\\
  $x_4$ & -2.8715 &0.1225\\
  $x_5$ & -6.2894 &0.1396\\
  $x_6$ & -10.7742 &0.1518\\
  $x_7$ & -16.4112 &0.1544\\
  $x_8$ & -23.2174 &0.1436\\
  $x_9$ & -31.1026 &0.1135\\
  $x_{10}$ & -39.8262 &0.0572\\
  \hline
\end{tabular}
\end{center}

$$h=\frac1{2^3}$$
\begin{center}
\begin{tabular}{|c|c|c|}
  \hline
  % after \\: \hline or \cline{col1-col2} \cline{col3-col4} ...
        & approx$f''(x)$ &error\\
  \hline
  $x_0$ & 2.9254 &0.0119\\
  $x_1$ &2.3904  &0.0157\\
  $x_2$ &1.3574  &0.0211\\
  $x_3$ &-0.3155  &0.0263\\
  $x_4$ &-2.7795  &0.0305\\
  $x_5$ &-6.1837  &0.0338\\
  $x_6$ &-10.6611  &0.0387\\
  $x_7$ &-16.2950  &0.0382\\
  $x_8$ &-23.1098  &0.0359\\
  $x_9$ &-31.0182  &0.0292\\
  $x_{10}$ &-39.7843  &0.0153\\
  \hline
\end{tabular}
\end{center}

$$h=\frac1{2^4}$$
\begin{center}
\begin{tabular}{|c|c|c|}
  \hline
  % after \\: \hline or \cline{col1-col2} \cline{col3-col4} ...
        & approx$f''(x)$ &error\\
  \hline
  $x_0$ &2.9363  &0.0011\\
  $x_1$ &2.4038 &0.0023\\
  $x_2$ &1.3722  &0.0063\\
  $x_3$ &-0.2970  &0.0077\\
  $x_4$ &-2.7571  &0.0081\\
  $x_5$ &-6.1568  &0.0069\\
  $x_6$ &-10.6317  &0.0092\\
  $x_7$ &-16.2637  &0.0068\\
  $x_8$ &-23.0835  &0.0097\\
  $x_9$ &-30.9965  &0.0075\\
  $x_{10}$ &-39.7747  &0.0057\\
  \hline
\end{tabular}
\end{center}

$$h=\frac1{2^5}$$
\begin{center}
\begin{tabular}{|c|c|c|}
  \hline
  % after \\: \hline or \cline{col1-col2} \cline{col3-col4} ...
        & approx$f''(x)$ &error\\
  \hline
  $x_0$ &2.9389  &-0.0015\\
  $x_1$ &2.4166  &-0.0105\\
  $x_2$ &1.3824  &-0.0039\\
  $x_3$ &-0.2970  &0.0077\\
  $x_4$ &-2.7443  &-0.0047\\
  $x_5$ &-6.1440  &-0.0059\\
  $x_6$ &-10.6291  &0.0067\\
  $x_7$ &-16.2509  &-0.0060\\
  $x_8$ &-23.0707  &-0.0031\\
  $x_9$ &-30.9965  &0.0075\\
  $x_{10}$ &-39.7824  &0.0133\\
  \hline
\end{tabular}
\end{center}
\newpage
$$h=\frac1{2^6}$$
\begin{center}
\begin{tabular}{|c|c|c|}
  \hline
  % after \\: \hline or \cline{col1-col2} \cline{col3-col4} ...
        & approx$f''(x)$ &error\\
  \hline
  $x_0$ &2.9491  &-0.0117\\
  $x_1$ &2.4166  &-0.0105\\
  $x_2$ &1.3517  &0.0268\\
  $x_3$ &-0.3277  &0.0384\\
  $x_4$ &-2.7443  &-0.0047\\
  $x_5$ &-6.1030  &-0.0468\\
  $x_6$ &-10.6496  &0.0271\\
  $x_7$ &-16.2202  &-0.0367\\
  $x_8$ &-23.0605  &-0.0134\\
  $x_9$ &-31.0067  &0.0177\\
  $x_{10}$ &-39.7722  &0.0031\\
  \hline
\end{tabular}
\end{center}
\section{}
Consider truncation error:
\begin{equation*}
  f''(x_0)=\frac{f(x+h)-2f(h)+f(x-h)}{h^2}+E(f,h)
\end{equation*}
\begin{equation*}
  E(f,h)=\frac{e_1-2e_2+e3}{h^2}-\frac{h^2f^{(4)}(\xi)}{12}
\end{equation*}
Assume $e_i\sim O(\epsilon)$
\begin{equation*}
  |E(f,h)|\leq\frac{4\epsilon}{h^2}+\frac{h^2M}{12}
\end{equation*}
$\epsilon=5\times10^{-5}, M=\max_{x\in[1,3]}f^{(4)}(x)=|f^{(4)}(\frac34\pi)|$\\
Mark $g(h)=\frac{4\epsilon}{h^2}+\frac{h^2M}{12}$\\
Take \begin{align*}
       &\frac{\partial g(h)}{\partial h}=0  \\
       \Leftrightarrow &\frac{Mh}{6}-\frac{8\epsilon}{h^3}=0\\
       \Leftrightarrow &h=(\frac{48\epsilon}{M})^{\frac{1}{4}}=0.0947
\end{align*}

\section{}

\subsection{Algorithm}
if we add the white noise $\epsilon\sim N(0,3^2)$ to $f(x_i)$, even if ignore the round-off error, the total error could be unacceptable.\\
In order to solve this problem:
\paragraph{STEP1} It is necessary to "clean" these original data. one of the appropriate method is one-dimensional Gaussian filter. \\
A Gaussian filter is a filter whose impulse response is a Gaussian function (or an approximation to it). Gaussian filters have the properties of having no overshoot to a step function input while minimizing the rise and fall time.\\
one-dimensional Gaussian filter has an impulse response given by
\begin{equation*}
  g(x)=\frac{1}{\sqrt(2\pi)\sigma}e ^{-\frac{x^2}{2\sigma ^2}}
\end{equation*}
and the frequency response is given by the Fourier transform
\begin{equation*}
  \hat g(f)=e^ {-\frac{f^2}{2\sigma^2_f}}
\end{equation*}
\paragraph{STEP2} Use the polynomial least square method to fit.
\subsection{Data insights}
After several experiments, it is proper is use 4-order polynomial.
\begin{figure}[h]
  \centering
  \includegraphics[width=1\textwidth,angle=0]{NE-1}\\
  \label{1}
\end{figure}
\\
By using the formula $f''(x)=\frac{f(x-H)-2f(x)+f(x+h)}{h^2}$ in the and take $h=\frac12$.
\begin{center}
\begin{tabular}{|c|c|c|}
  \hline
  % after \\: \hline or \cline{col1-col2} \cline{col3-col4} ...
        & approx$f''(x)$ &error\\
  \hline
  $x_0$ &-2.1122  &5.0496\\
  $x_1$ &-1.2505  &3.6566\\
  $x_2$ &-1.3645  &2.7430\\
  $x_3$ &-2.4541  &2.1648\\
  $x_4$ &-4.5193  &1.7703\\
  $x_5$ &-7.5602  &1.4104\\
  $x_6$ &-11.5768 &0.9543\\
  $x_7$ &-16.5690  &0.3122\\
  $x_8$ &-22.5369 &-0.5370\\
  $x_9$ &-29.4804  &-1.5087\\
  $x_{10}$ &-37.3995  &-2.3695\\
  \hline
\end{tabular}
\end{center}
\subsection{Optimization}
Clearly, the error is still relative large. But without the interruption of white noise as well as according to the $f(x)=e^x\sin x$ is infinitely differentiable, Richardson extrapolation is a great approach to improve precision.
For instance, extrapolate 1-order, and get a new formula which is
\begin{equation*}
f''(x)=\frac{-f(x+2h)+16f(x+h)-30f(x)+16f(x-h)-f(x-2h)}{12h^2}+E(f,h).
\end{equation*}
Similarly, let $\varepsilon$ represent the order of error, then
\begin{equation*}
  |E(f,h)|=\frac{16\varepsilon}{3h^2}+\frac{h^4f^{(6)}(\xi)}{90}
\end{equation*}
The result of using highly precise formula is
\begin{center}
\begin{tabular}{|c|c|c|}
  \hline
  % after \\: \hline or \cline{col1-col2} \cline{col3-col4} ...
        & approx$f''(x)$ &error\\
  \hline
  $x_0$ &-1.6040 &4.5414\\
  $x_1$ &-0.7423 &3.1485\\
  $x_2$ &-0.8563  &2.2348\\
  $x_3$ &-1.9459  &1.6567\\
  $x_4$ &-4.0112  &1.2622\\
  $x_5$ &-7.0521  &0.9022\\
  $x_6$ &-11.0686 &0.4462\\
  $x_7$ &-16.0608  &-0.1960\\
  $x_8$ &-22.0287 &-1.0451\\
  $x_9$ &-28.9722  &-2.0168\\
  $x_{10}$ &-36.8914  &-2.8777\\
  \hline
\end{tabular}
\end{center}
As you can see, the error is smaller.
\subsection{Strength and Weakness}

\subsubsection{Strength}
\paragraph{(1)} In theory, extrapolation could be used infinitely, that is, the approximated result incline to precise result.
\subsubsection{Weakness}
\paragraph{(1)} Gaussian filter is a statistical algorithm, without comparing with precise result, it is not easy to do acceptable "cleaning".
\paragraph{(2)} In some cases, for example $|f'(x)|$ turn out to be large, the result Gaussian filter will be worse.
\\
\begin{figure}[h]
  \centering
  \includegraphics[width=1\textwidth,angle=0]{NE-2}\\
  \label{2}
\end{figure}
\\
\paragraph{(3)} In fact, it is impossible to know the mathematical expression of $f(x)$. In other words, the Differentiability is unknown. we are not able to estimate error.
\newpage
\section{Appendix}
\begin{lstlisting}
function [d2x,err]=secdiff(h)
d2x=zeros(11,1);
x=zeros(11,1);
for i=1:11
    x(i)=1+(2*(i-1))./10;
end
err=zeros(11,1);
function s=f(x)
s=exp(x).*sin(x);
end
function s=g(x)
s=2.*exp(x).*cos(x);
end
for i=1:11
    d2x(i)=(f(x(i)+h)-2.*f(x(i))+f(x(i)-h))./(h.^2);
end
err(1:11)=g(x(1:11))-d2x(1:11);
end


function y_filted = Gaussianfilter(r, sigma, y)
GaussTemp = ones(1,r*2-1);
for i=1 : r*2-1
    GaussTemp(i) = exp(-(i-r)^2/(2*sigma^2))/(sigma*sqrt(2*pi));
end
y_filted = y;
for i = r : length(y)-r+1
    y_filted(i) = y(i-r+1 : i+r-1)*GaussTemp';
end


function [d2x,err]=ylmq(h,f)
d2x=zeros(11,1);
x=zeros(11,1);
for i=1:11
    x(i)=1+(2*(i-1))./10;
end
err=zeros(11,1);
function s=g(x)
s=2.*exp(x).*cos(x);
end
for i=1:11
    d2x(i)=(-feval(f,x(i)+2.*h)+16.*feval(f,x(i)+h)-30.*feval(f,x(i))+16.*feval(f,x(i)-h)-feval(f,x(i)-2.*h))./(12.*h.^2);
end
err(1:11)=g(x(1:11))-d2x(1:11);
end


xx=1:0.05:3
yy=fff(xx)
noise=randn(1,length(yy)).*0.5
yyNoise=yy+noise
yyfilter=Gaussianfilter(3,1,yyNoise)
plot(xx,yy);
hold on;
plot(xx,yyNoise);
hold on;
plot(xx,yyfilter);
grid on
\end{lstlisting}

\end{document}
